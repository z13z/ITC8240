\documentclass[12pt, a4paper]{article}
\usepackage[utf8]{inputenc}

\title{Homework 1}
\author{Zaza Gogia}

\begin{document}

\maketitle

\section*{Task 1}
Our result ciphertext is \textbf{QMNDECPJEK}.\\\\
I have written python 3 code to solve this problem.\\ link for code is:  https://github.com/z13z/ITC8240/blob/main/HW1/Task1.py \\
Code is quite simple and self explanatory
I have function, called "encrypt\_cezar\_letter", that calculates letter after shifting with given key. shifted letter is calculated like this
index in alphabet (not character) of our new letter is:\\ (our letter character index - index of 'A' + shift) mod 26\\ if we add index of character 'A' to calculated number we will get character index of shifted letter.\\ In method "encrypt\_cezar" we
just call function "encrypt\_cezar\_letter" for every letter.\\
For shuffle encryption I have method called "encrypt\_shuffle" it just takes message copies it creates new list and inserts letter corresponding to given shuffle array. New letter index in original message is calculated like this:
current number in shuffle array + (how many times did we finish key length iterations) * (shuffle array length).\\
After we got this functions working, getting ciphertext is very easy: \\
get $S_1$ with encrypt\_cezar('BLOCKCHAIN', 9), give generated $S_1$ to function encrypt\_shuffle($S_1$, $K_{P_!}$) to calculate $P_1$, than call encrypt\_cezar($P_1$, $S_2$) to get $S_2$, and after this one I call encrypt\_shuffle($S_2$, $K_{P_2}$ to get final cyphertext

\section*{Task 2}
\subsection*1{Ciphertext is: \textbf{QZAXYLKFLSWMSMOHCALXYMEBPA}}
\subsection*2{Index of coincidence of the plaintext is: \textbf{0.070769}}
\subsection*3{Index of coincidence of the ciphertext is: \textbf{0.036923}}

\end{document}
