\documentclass[12pt, a4paper]{article}
\usepackage[utf8]{inputenc}
\usepackage{amsmath}
\usepackage{hyperref}
\hypersetup{
    colorlinks=true,
    urlcolor=cyan,
    }

\title{Homework 1}
\author{Zaza Gogia}

\begin{document}

\maketitle

\section*{Task 1}
Our result ciphertext is \textbf{QMNDECPJEK}.\\\\
I have written python 3 code to solve this problem.\\ link for code is on \href[]{https://github.com/z13z/ITC8240/blob/main/HW1/Task1.py}{github} \\
Code is quite simple and self explanatory
I have function, called "encrypt\_cezar\_letter", that calculates letter after shifting with given key. shifted letter is calculated like this
index in alphabet (not character) of our new letter is:\\ (our letter character index - index of 'A' + shift) mod 26\\ if we add index of character 'A' to calculated number we will get character index of shifted letter.\\ In method "encrypt\_cezar" we
just call function "encrypt\_cezar\_letter" for every letter.\\
For shuffle encryption I have method called "encrypt\_shuffle" it just takes message copies it creates new list and inserts letter corresponding to given shuffle array. New letter index in original message is calculated like this:
current number in shuffle array + (how many times did we finish key length iterations) * (shuffle array length).\\
After we got this functions working, getting ciphertext is very easy: \\
get $S_1$ with encrypt\_cezar('BLOCKCHAIN', 9), give generated $S_1$ to function encrypt\_shuffle($S_1$, $K_{P_!}$) to calculate $P_1$, than call encrypt\_cezar($P_1$, $S_2$) to get $S_2$, and after this one I call encrypt\_shuffle($S_2$, $K_{P_2}$ to get final cyphertext

\section*{Task 2}
\subsection*1{Ciphertext is: \textbf{QZAXYLKFLSWMSMOHCALXYMEBPA}}
\subsection*2{Index of coincidence of the plaintext is: \textbf{0.070769}}
\subsection*3{Index of coincidence of the ciphertext is: \textbf{0.036923}}\\\\\\
I have written python 3 code to solve this problem.\\ link for code is on
\href{https://github.com/z13z/ITC8240/blob/main/HW1/Task2.py}{github} \\
Main function for encryption is "encrypt\_vigenere", it cycles through the given plaintext and calls function "encrypt\_cezar\_letter" (same one that I used and explained in Task 1). I calculate shift number from key with formula: shift number = current key letter character index - character index for letter 'A'. After I call this function for each letter I have encrypted text.\\Infinal step I calculate index of coincidence with my second main function called "get\_index\_of\_coincidence". It gets dictionary of characters and their counts in plain text using function "get\_letters\_count\_dict", using this dictionary I calculate $\sum(Val_i*(Val_{i-1}))$ and divide result by (text length * (text length - 1)). After that I return result which is number of coincidence for given text. I used that function to calculate second and third subsection of given task.

\section*{Task 3}
\subsection*1{}
Let's translate both ciphertext and plaintext to numbers:
SURFACE $\rightarrow$ 18 20 17 5 0 2 4\\
NJCAXTP $\rightarrow$ 13 9 2 0 23 19 15\\
So we have function that crosses points (18, 13); (20, 9); (17, 2); (5, 0); (0, 23); (2, 19); (4, 15).
Lets find encryption function that satisfies all the equations:\\
$\begin{cases}$13 = a * 18 + b mod 26$\\$9 = a * 20 + b mod 26$\\$2 = a * 17 + b mod 26$\\...\end{cases}$\\
for solving equations I use python3 script that is uploaded on \href{https://github.com/z13z/ITC8240/blob/main/HW1/Task3.py}{github}\\
After solving this equation I got encryption key $f(x) = 11x + 23$

\subsection*2{}
NJCAXTP $\rightarrow$ 13 9 2 0 23 19 15\\
SURFACE $\rightarrow$ 18 20 17 5 0 2 4\\
So we have function that crosses points (13, 18); (9, 20); (2, 17); (0, 5); (23, 0); (19, 2); (15, 4).
Lets find decryption function that satisfies all the equations:\\
$\begin{cases}$18 = a * 13 + b mod 26$\\$20 = a * 9 + b mod 26$\\$17 = a * 2 + b mod 26$\\...\end{cases}$\\
for solving equations I use python3 script that is uploaded on \href{https://github.com/z13z/ITC8240/blob/main/HW1/Task3.py}{github}\\
After solving this equation I got encryption key $f(x) = 19x + 5$
\end{document}
